\documentclass[UTF8,a4paper,12pt]{ctexbook} 

 \usepackage{graphicx}%学习插入图
 \usepackage{verbatim}%学习注释多行
 \usepackage{booktabs}%表格
 \usepackage{geometry}%图片
 \usepackage{amsmath}
 \usepackage{amssymb}
 \usepackage{listings}%代码
 \usepackage{xcolor}  %颜色
 \usepackage{enumitem}%列表格式
 \usepackage{tcolorbox}
 \usepackage{algorithm}  %format of the algorithm
 \usepackage{algorithmic}%format of the algorithm
 \usepackage{multirow}   %multirow for format of table
 \usepackage{tabularx} 	%表格排版格式控制
 \usepackage{array}	%表格排版格式控制
 \usepackage{hyperref} %超链接 \url{URL}
 % \setCJKmainfont{方正兰亭黑简体}  %中文字体设置
 % \setCJKsansfont{华康少女字体} %设置中文字体
 % \setCJKmonofont{华康少女字体} %设置中文字体

 \CTEXsetup[format+={\flushleft}]{section}


 
 %%%% 下面的命令定义图表、算法、公式 %%%%
 \newcommand{\EQ}[1]{$\textbf{EQ:}#1\ $}
 \newcommand{\ALGORITHM}[1]{$\textbf{Algorithm:}#1\ $}
 \newcommand{\Figure}[1]{$\textbf{Figure }#1\ $}
 
 %%%% 下面命令改变图表下标题的前缀 %%%%% 如:图-1、Fig-1
 \renewcommand{\figurename}{Fig}
 
 \geometry{left=1.6cm,right=1.8cm,top=2cm,bottom=1.7cm} %设置文章宽度
 
 \pagestyle{plain} 		  %设置页面布局

 %代码效果定义
 \definecolor{mygreen}{rgb}{0,0.6,0}
 \definecolor{mygray}{rgb}{0.5,0.5,0.5}
 \definecolor{mymauve}{rgb}{0.58,0,0.82}
 \lstset{ %
 	backgroundcolor=\color{white},   % choose the background color
 	basicstyle=\footnotesize\ttfamily,      % size of fonts used for the code
 	%stringstyle=\color{codepurple},
 	%basicstyle=\footnotesize,
 	%breakatwhitespace=false,         
 	%breaklines=true,                 
 	%captionpos=b,                    
 	%keepspaces=true,                 
 	%numbers=left,                    
 	%numbersep=5pt,                  
 	%showspaces=false,                
 	%showstringspaces=false,
 	%showtabs=false,        
 	columns=fullflexible,
 	breaklines=true,                 % automatic line breaking only at whitespace
 	captionpos=b,                    % sets the caption-position to bottom
 	tabsize=4,
 	commentstyle=\color{mygreen},    % comment style
 	escapeinside={\%*}{*)},          % if you want to add LaTeX within your code
 	keywordstyle=\color{blue},       % keyword style
 	stringstyle=\color{mymauve}\ttfamily,     % string literal style
 	frame=single,
 	rulesepcolor=\color{red!20!green!20!blue!20},
 	% identifierstyle=\color{red},
 	language=c++,
 }
 \author{\kaishu 郑华}
 \title{\heiti Redis笔记}
 
\begin{document}          %正文排版开始
 	\maketitle
  

\chapter{Redis 简介}
	
	\url{http://www.runoob.com/redis/redis-backup.html}
	
	\section{应用场景}
		\begin{itemize}
			\item 缓存
			\item 聊天室,秒杀,任务队列
			\item 应用排行榜.网站统计
			\item 数据存储(add,del,update,select)定期持久化到硬盘中
			\item 分布式集群架构中的session 分离
		\end{itemize}
	\section{安装与启动}
		\begin{lstlisting}
	// 安装
	$ wget http://download.redis.io/releases/redis-2.8.17.tar.gz
	$ tar xzf redis-2.8.17.tar.gz
	$ cd redis-2.8.17
	$ make	
	
	
	//启动redis服务.
	$ cd src
	$ ./redis-server
	
	//使用默认配置启动redis	服务
	./redis-server redis.conf
	
	//启动redis服务进程后,就可以使用测试客户端程序redis-cli和redis服务交互了	
	./redis-cli		
		\end{lstlisting}

\chapter{基本操作}	
	
1	\section{Key}
		\subparagraph{须知}redis命令\textbf{不区分大小写},所以\verb|get var|和\verb|GET var|是等价的
		
		\subparagraph{keys [pattern]} 返回相应的的key
		
		\subparagraph{dump key } 序列化给定key,并返回序列化的值
		
		\subparagraph{randomkey} 返回随机的key
		
		\subparagraph{exists key} 判断key是否存在
		
		\subparagraph{type key} 返回key 存储的类型
		
		\subparagraph{del key} 删除key
		
		\subparagraph{expire key seconds} 给key设置过期时间
		
		\subparagraph{persist key} 移除key 的过期时间,key将持久保存
		
	\section{SET}\verb|SET name "runoob"|->\verb|SET key value|
		
	\section{GET}\verb|GET key|
	

\chapter{数据类型}	
	\section{string}
		\textbf{二进制安全的}。意思是redis的string可以包含任何数据。比如jpg图片或者序列化的对象 
		
		一个键最大能存储\verb|512MB|
		
	\section{list}
	
	\section{hash}
	
	\section{set}
	
	\section{zset}
	
	\section{事务与锁}
		类似于mysql 的 start transaction, 可以保证原子性,可以使用rollback 取消操作。
		
		redis 使用 multi  命令实现
		
		使用 watch 锁,解决多用户竞争

\chapter{持久化}	
	\section{rdb 持久化}
	
	
	\section{aof 持久化}
	

\chapter{分布式}	
	\section{主从复制}
	
	\section{分布式集群}

\chapter{应用}	
	\section{位图法统计活跃用户}
		\begin{itemize}
			\item 1亿个用户
			\item 如何记录用户的登录信息
			\item 如何查询活跃用户,一周连续登录
		\end{itemize}
	
		从信息的角度看,一个用户的登录只需要 一个位 就可以表示。 0-1
		
		在数据库中,数据一般都有编号
		
		简化如:
			\begin{lstlisting}
	7个用户
	
	周一 01011100
	周二 01010010
	周三 01111000
	
	setbit  mon   100000000  0  // 初始化为0
	setbit  mon   3     1       // 第3号用户登录,标记为1
	setbit  mon   9     1
	
	
	setbit  tuseday 10000000 0
	set bit tuesday 4  1
	

	bitop and mon tuseday ...
	
			\end{lstlisting}
	\section{频道发布与订阅}
	
	\section{微博之用户注册与微博发布}
	
	\section{微博之粉丝关系与推送微博}
	
	\section{哈希数据存储微博}
		    
\end{document} 
 		    