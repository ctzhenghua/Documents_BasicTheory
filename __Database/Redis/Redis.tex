\documentclass[UTF8,a4paper,12pt]{ctexbook} 

 \usepackage{graphicx}%学习插入图
 \usepackage{verbatim}%学习注释多行
 \usepackage{booktabs}%表格
 \usepackage{geometry}%图片
 \usepackage{amsmath}
 \usepackage{amssymb}
 \usepackage{listings}%代码
 \usepackage{xcolor}  %颜色
 \usepackage{enumitem}%列表格式
 \usepackage{tcolorbox}
 \usepackage{algorithm}  %format of the algorithm
 \usepackage{algorithmic}%format of the algorithm
 \usepackage{multirow}   %multirow for format of table
 \usepackage{tabularx} 	%表格排版格式控制
 \usepackage{array}	%表格排版格式控制
 \usepackage{hyperref} %超链接 \url{URL}
 % \setCJKmainfont{方正兰亭黑简体}  %中文字体设置
 % \setCJKsansfont{华康少女字体} %设置中文字体
 % \setCJKmonofont{华康少女字体} %设置中文字体

 \CTEXsetup[format+={\flushleft}]{section}


 
 %%%% 下面的命令定义图表、算法、公式 %%%%
 \newcommand{\EQ}[1]{$\textbf{EQ:}#1\ $}
 \newcommand{\ALGORITHM}[1]{$\textbf{Algorithm:}#1\ $}
 \newcommand{\Figure}[1]{$\textbf{Figure }#1\ $}
 
 %%%% 下面命令改变图表下标题的前缀 %%%%% 如:图-1、Fig-1
 \renewcommand{\figurename}{Fig}
 
 \geometry{left=1.6cm,right=1.8cm,top=2cm,bottom=1.7cm} %设置文章宽度
 
 \pagestyle{plain} 		  %设置页面布局

 %代码效果定义
 \definecolor{mygreen}{rgb}{0,0.6,0}
 \definecolor{mygray}{rgb}{0.5,0.5,0.5}
 \definecolor{mymauve}{rgb}{0.58,0,0.82}
 \lstset{ %
 	backgroundcolor=\color{white},   % choose the background color
 	basicstyle=\footnotesize\ttfamily,      % size of fonts used for the code
 	%stringstyle=\color{codepurple},
 	%basicstyle=\footnotesize,
 	%breakatwhitespace=false,         
 	%breaklines=true,                 
 	%captionpos=b,                    
 	%keepspaces=true,                 
 	%numbers=left,                    
 	%numbersep=5pt,                  
 	%showspaces=false,                
 	%showstringspaces=false,
 	%showtabs=false,        
 	columns=fullflexible,
 	breaklines=true,                 % automatic line breaking only at whitespace
 	captionpos=b,                    % sets the caption-position to bottom
 	tabsize=4,
 	commentstyle=\color{mygreen},    % comment style
 	escapeinside={\%*}{*)},          % if you want to add LaTeX within your code
 	keywordstyle=\color{blue},       % keyword style
 	stringstyle=\color{mymauve}\ttfamily,     % string literal style
 	frame=L,
 	xleftmargin = .05\textwidth,
 	rulesepcolor=\color{red!20!green!20!blue!20},
 	% identifierstyle=\color{red},
 	language=php,
 }
 \author{\kaishu 郑华}
 \title{\heiti Redis笔记}
 
\begin{document}          %正文排版开始
 	\maketitle
  

\chapter{Redis 简介}
	
	\section{参考文献}
		\url{http://www.runoob.com/redis/redis-backup.html}
	
		进阶:\url{https://www.cnblogs.com/hjwublog/p/5660578.html}
	\section{应用场景}
		\begin{itemize}
			\item 缓存
			\item 聊天室,秒杀,任务队列
			\item 应用排行榜.网站统计
			\item 数据存储(add,del,update,select)定期持久化到硬盘中
			\item 分布式集群架构中的session 分离
		\end{itemize}
	\section{安装与启动}
		\begin{lstlisting}
	// 安装
	$ wget http://download.redis.io/releases/redis-2.8.17.tar.gz
	$ tar xzf redis-2.8.17.tar.gz
	$ cd redis-2.8.17
	$ make	
	
	
	//启动redis服务.
	$ cd src
	$ ./redis-server
	
	//使用默认配置启动redis	服务
	./redis-server redis.conf
	
	//启动redis服务进程后,就可以使用测试客户端程序redis-cli和redis服务交互了	
	./redis-cli		
		\end{lstlisting}

\chapter{基本操作}	
	redis命令\textbf{不区分大小写},所以\verb|get var|和\verb|GET var|是等价的
	
	\section{选库}
		使用\textbf{Select} 命令用于\textit{切换到}\textbf{指定的数据库},数据库索引号 index 用数字值指定,以 0 作为起始索引值。
	
	\section{Key}
		\begin{table}[H]
			\centering
			\caption{常用命令}
			\begin{tabular}{p{5cm}<{\centering} | p{10cm}<{\centering}}
				\hline
					命令  &  含义 \\
				\hline
					\textbf{keys} [pattern] & 返回相应的的key \\
					\textbf{set} key value & 设定一个key-value \\
					\textbf{get} key &  获取一个key的值\\
					\textbf{dump} key & 序列化给定key,并返回序列化的值\\
					\textbf{randomkey} & 返回随机的key \\
					\textbf{exists} key & 判断key是否存在 \\
					\textbf{type} key & 返回key 存储的类型 \\
					\textbf{del} key &  删除key \\
					\textbf{expire} key seconds & 给key设置过期时间 \\
					\textbf{persist} key & 移除key 的过期时间,key将持久保存 \\
				\hline
			\end{tabular}
		\end{table}

	
	
\chapter{数据类型}	
	\section{string}
		\textbf{二进制安全的}。意思是redis的string可以包含任何数据。比如jpg图片或者序列化的对象 
		
		一个键最大能存储\verb|512MB|
		
		\begin{itemize}[itemindent = 1em] 
			\item \verb@set key value [ex 秒数]|[px 毫秒数]  [nx]|[xx]@	:设置键值
				\begin{itemize}
					\item \verb|nx |表示key 不存在时 执行操作
					\item \verb|xx |表示key 存在时 执行操作
					\item \verb|ex |表示设置 过期时间
					\item \verb|px |表示设置 持续时间,\verb|ex |与\verb|px |不能同时设置
				\end{itemize}
			\item \verb|mset key1 v1 key2 v2 ...| :一次性设置多个键值对(multi set) 			
			\item \verb|get key | 获取key 的值
			\item \verb|mget key1 key2 ...| :一次性获取多个键的值
			\item \verb|setrange key offset value |:把字符串的\verb|offset |偏移字节改为\verb|value|
				\begin{lstlisting}
	set greet hello
	get greet --> hello
	setrange greet 2 x
	get greet --> hexlo
	
	setrange greet 2 ??
	get greet --> he??o
	
	// 当偏移量大于字符长度时,多余部分将自动以0x00 填充
	setrange greet 6 !
	get greet --> he??o0x00!
				\end{lstlisting}
			\item \verb|append key value| :把value 追加到key 的原值上
			\item \verb|getrange key start stop| :获取字符串中 \verb|start, stop| 范围的值
				\begin{itemize}
					\item 左闭右闭区间,从0开始
					\item 负数 表示倒数第多少
				\end{itemize}
			
			\item \verb|getset key newValue| :获取并返回旧值,并设置新值
			\item \verb@incr|decr key@ :增1或减1,不存在的key 当成0再incr 返回。与之对应的有\verb|incrby key num|
			\item \verb|getbit key offset| :获取值的二进制的对应位上的值,从高位开始
			\item \verb|setbit key offset value| :设置对应2进制位上的值,offset 最长能达到\verb|2^32-1|位
			\item \verb|bitop op destKey key1 [key2 key3 ...]| :对[key1 key2..] 做op,并将结果保存到destKey 中,op 可以为一下几种\verb|AND OR NOT XOR|
		\end{itemize}
				
	\section{list}
		按照插入顺序排序。你可以添加一个元素到列表的头部(左边)或者尾部(右边)
		
		\begin{table}[H]
			\centering
			\caption{list操作说明}
			\begin{tabular}{p{5cm}<{\centering} | p{10cm}<{\centering}}
				\toprule
					命令 &  含义 \\
				\midrule
					\textbf{lpush} key value1[value2] &  	\\
					\textbf{lpop} key & 		\\
					\textbf{blpop} key1[key2] timeout &		\\
					\textbf{brpop} key1[key2] timeout &		\\
					\textbf{lindex} key index &		\\
					\textbf{linsert}  &		\\
				    \textbf{llen} key & 		\\
					\textbf{lpushx} key value & 		\\
					\textbf{lrange} key start stop & 		\\
					\textbf{lset} key index value &		\\
					\textbf{ltrim} key start stop & 		\\
					\textbf{rpop} key & 		\\
					\textbf{rpoplpush}  & 		\\
					\textbf{rpush} key value1[value2] & 		\\
					\textbf{rpushx} key value & 		\\
				\bottomrule
			\end{tabular}
		\end{table}
		
	\section{hash}
		\begin{table}[H]
			\centering
			\caption{hash 操作说明}
			\begin{tabular}{p{5cm}<{\centering} | p{10cm}<{\centering}}
				\toprule
					命令 &  含义 \\
				\midrule
					\textbf{hdel} key field[field2]&  	\\
					\textbf{hexists} key filed & 		\\
					\textbf{hget} key field &			\\
					\textbf{hincrby} key field increment &		\\
					\textbf{hkeys} key &					\\
					\textbf{hlen} key &					\\
					\textbf{hmget} key field1[filed2] & 	\\
					\textbf{hmset} key field1[field2] & 	\\
					\textbf{hset} key field value & 		\\
					\textbf{hsetnx} key field value &	\\
					\textbf{hvals} key & 				\\
				\bottomrule
			\end{tabular}
		\end{table}
		
	\section{set}
		Set 是 String 类型的无序集合。集合成员是唯一的,这就意味着集合中不能出现重复的数据。
		
		Redis 中集合是通过哈希表实现的,所以添加,删除,查找的复杂度都是 O(1)。
		
		\begin{table}[H]
			\centering
			\caption{set操作说明}
			\begin{tabular}{p{5cm}<{\centering} | p{10cm}<{\centering}}
				\toprule
					命令 &  含义 \\
				\midrule
				\textbf{SADD} key mem1 [m2] 	 &  	\\
				\textbf{SCARD} key 	 & 		\\
				\textbf{SDIFF} key1 [key2] 	 &		\\
				\textbf{SINTER} key1 [key2] 	 &		\\
				\textbf{SISMEMBER} key mem 	 &		\\
				\textbf{SMEMBERS} key 	 &		\\
				\textbf{SMOVE} sr des mem 	 & 		\\
				\textbf{SPOP} key 	 & 		\\
				\textbf{SRANDMEMBER} key	 & 		\\
				\textbf{SREM} key mem1 [mem2] 	 &		\\
				\textbf{SUNION} key1 [key2] 	 & 		\\
				\bottomrule
			\end{tabular}
		\end{table}	
		
	\section{zset}
		有序集合和集合一样也是string类型元素的集合,且不允许重复的成员。
		
		不同的是每个元素都会关联一个double类型的分数。redis正是通过分数来为集合中的成员进行从小到大的排序。
		
		有序集合的成员是唯一的,但分数(score)却可以重复。
		
		集合是通过哈希表实现的,所以添加,删除,查找的复杂度都是O(1)。
		\begin{table}[H]
			\centering
			\caption{zset操作说明}
			\begin{tabular}{p{5cm}<{\centering} | p{11cm}<{\centering}}
				\toprule
					命令 &  含义 \\
				\midrule
				\textbf{ZADD}    &  	\\
				\textbf{ZCARD}   & 		\\
				\textbf{ZCOUNT}	 &		\\
				\textbf{ZINCRBY} &		\\
				\textbf{ZRANGE}	 &		\\
				\textbf{ZRANGEBYLEX}   &		\\
				\textbf{ZRANGEBYSCORE} & 		\\
				\textbf{ZRANK}  & 		\\
				\textbf{ZREM} 	& 		\\
				\textbf{ZREMRANGEBYLEX} &		\\
				\textbf{ZREVRANK} 	    & 		\\
				\bottomrule
			\end{tabular}
		\end{table}

	\section{事务与锁}
		类似于mysql 的 start transaction, 可以保证原子性,可以使用rollback 取消操作。
		
		redis 使用 multi  命令实现
		
		使用 watch 锁,解决多用户竞争
		
		Redis 事务可以一次执行多个命令, 并且带有以下两个重要的保证:
			\begin{itemize}[itemindent = 2em]
				\item 批量操作在发送 EXEC 命令前被放入队列缓存。
				\item 收到 EXEC 命令后进入事务执行,事务中任意命令执行失败,其余的命令依然被执行。
				\item 在事务执行过程,其他客户端提交的命令请求不会插入到事务执行命令序列中。
			\end{itemize}
		
		一个事务从开始到执行会经历以下三个阶段:
			\begin{enumerate}[itemindent = 2em]
				\item 开始事务。
				\item 命令入队。
				\item 执行事务。
			\end{enumerate}
		
		\begin{table}[H]
			\centering
			\caption{事务与锁 操作说明}
			\begin{tabular}{p{5cm}<{\centering} | p{10cm}<{\centering}}
				\toprule
					命令 &  含义 \\
				\midrule
				\textbf{MULTI} 	 &  标记一个事务块的开始。	\\
				\textbf{EXEC} 	 & 	执行所有事务块内的命令。	\\
				\textbf{DISCARD} 	 &	取消事务,放弃执行事务块内的所有命令。	\\
				\textbf{UNWATCH} 	 &	取消 WATCH 命令对所有 key 的监视。	\\
				\textbf{WATCH} key [key ...] 	 &	监视一个(或多个) key ,如果在事务执行之前这个(或这些) key 被其他命令所改动,那么事务将被打断。	\\
				\bottomrule
			\end{tabular}
		\end{table}

\chapter{订阅与发布}



\chapter{持久化}	
	\section{rdb 持久化}
		\paragraph{SAVE|BGSAVE}
			用于创建当前数据库的备份。
			
		\paragraph{恢复数据}
			如果需要恢复数据,只需将备份文件 (dump.rdb) \textbf{移动到} \verb|redis| \textbf{安装目录}并启动服务即可。
			
			获取 redis 目录可以使用 CONFIG 命令:\verb|CONFIG GET dir|
	
	\section{aof 持久化}
	

\chapter{分布式}	
	\section{主从复制}
	
	\section{分布式集群}
		\url{https://www.cnblogs.com/yuanermen/p/5717885.html}
		
		\url{https://www.cnblogs.com/cjsblog/p/9048545.html}
		
		
		\url{https://www.cnblogs.com/liyasong/p/redis_jiqun.html?utm_source=itdadao&utm_medium=referral}
		
	\section{Redis分区}
		分区是分割数据到多个Redis实例的处理过程,因此每个实例只保存key的一个子集。

		\paragraph{分区的优势}
			\begin{itemize}
				\item 通过利用多台计算机内存的和值,允许我们构造更大的数据库。
				\item 通过多核和多台计算机,允许我们扩展计算能力;通过多台计算机和网络适配器,允许我们扩展网络带宽。
			\end{itemize}
		
		redis集群大多数支持在\textbf{运行时}\textit{增加、删除节点}的透明数据平衡的能力,但是类似于客户端分区、代理等其他系统则不支持这项特性。然而,一种叫做\verb|presharding|的技术对此是有帮助的。
		
		\paragraph{分区类型}
			\begin{itemize}
				\item 范围分区:0-1000,1001-2000,..
				\item 哈希分区
			\end{itemize}
			
\chapter{应用}	
	\section{位图法统计活跃用户}
		\begin{itemize}
			\item 1亿个用户
			\item 如何记录用户的登录信息
			\item 如何查询活跃用户,一周连续登录
		\end{itemize}
	
		从信息的角度看,一个用户的登录只需要 一个位 就可以表示。 0-1
		
		在数据库中,数据一般都有编号
		
		简化如:
			\begin{lstlisting}
	7个用户
	
	周一 01011100
	周二 01010010
	周三 01111000
	
	setbit  mon   100000000  0  // 初始化为0
	setbit  mon   3     1       // 第3号用户登录,标记为1
	setbit  mon   9     1
	
	
	setbit  tuseday 10000000 0
	set bit tuesday 4  1
	

	bitop and mon tuseday ...
	
			\end{lstlisting}
	\section{频道发布与订阅}
	
	\section{微博之用户注册与微博发布}
	
	\section{微博之粉丝关系与推送微博}
	
	\section{哈希数据存储微博}



\chapter{PHP 与 Redis}
	\section{关联}
		\begin{lstlisting}
	$redis = new redis();  
	$result = $redis->connect('127.0.0.1', 6379);  
	var_dump($result); //结果:bool(true)  
	
	$result = $redis->set('test',"11111111111");  
	var_dump($result);    //结果:bool(true)  
	
	$result = $redis->get('test');  
	var_dump($result);   //结果:string(11) "11111111111"  
	
	$redis->delete('test');  
	var_dump($redis->get('test'));  //结果:bool(false)  
		\end{lstlisting}		    
\end{document} 
 		    